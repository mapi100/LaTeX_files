\documentclass[12pt]{article}
\usepackage[a4paper, total={5.5in, 9in}]{geometry}
\usepackage{amsmath}
\usepackage{amsfonts}
\usepackage{graphicx}
\usepackage{pgfplots}
\pgfplotsset{compat=1.18}
\usepackage{enumitem}
\usepackage{hyperref}

\title{Math for Business \& Social Science\\ Worksheet 9}
\author{PCL Learning Center}
\date{}

\begin{document}
\maketitle

\section*{Problem Set 1\\Difficulty level: Normal}
\section*{Problem 1}
Rewrite the expression as sum or difference of logarithms: 
\[
\log\left(\frac{nm^2}{px^2}\right)
\]

\section*{Problem 2}
Rewrite the expression as a single logarithm:
\begin{enumerate}[label=(\alph*)]
    \item \( a \cdot \log_3(x^3) - 2 \cdot \log_3(x^{-4}) + 5 \cdot \log_3(x) \)
    \item \( 5\log_8(x^3) + 2\log_8(x^{-4}) - 3\log_8(x) \)
    \item \( \ln(x - 1) + \ln(3) - 3\ln(x) \)
\end{enumerate}

\section*{Problem 3}
Find the value of \( x \) that satisfies the equation:
\[
\log_5(x + 10) - \log_5(x + 6) = \log_5(3)
\]

\section*{Problem 4}
In statistics, the sample regression equation \( y = ab^x \) is reduced to a linear form by taking logarithms of both sides. Express \( \log y \) in terms of \( x \), \( \log a \), and \( \log b \), and explain what is meant by saying that the resulting expression is linear.

\section*{Problem 5}
A manufacturer’s supply equation is:
\[
p = \log\left(10 + \frac{q}{2} \right)
\]
Where \( q \) is the number of units supplied at price \( p \) per unit. At what price will the manufacturer supply 3240 units?

\section*{Problem 6}
Expand the logarithmic expression as a sum or difference:
\[
\ln\left(\frac{4x^3}{(x + 2)^2} \right)
\]

\section*{Problem 7}
Combine the expression into a single logarithm:
\[
3 \ln(x) + \ln(2) - \ln(x^2 + 1)
\]

\section*{Problem 8}
Solve for \( x \):
\[
\log(x + 1) + \log(x - 1) = 1
\]
Hint: Use logarithmic properties and solve the resulting quadratic.


\section*{Problem Set 2\\Difficulty level: Hard}
\section*{Problem 1}
A bacterial culture grows according to the model:
\[
N(t) = N_0 e^{kt}
\]
where \( N_0 = 500 \) is the initial number of bacteria, and \( N(t) = 8000 \) after 6 hours.

\begin{itemize}
    \item[(a)] Find the value of \( k \) (round to 4 decimal places).
    \item[(b)] Use the value of \( k \) to determine when the population will reach 20,000. Round to the nearest tenth of an hour.
\end{itemize}


\end{document}
