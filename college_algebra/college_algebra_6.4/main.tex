\documentclass[12pt]{article}
\usepackage[a4paper, total={5.5in, 9in}]{geometry}
\usepackage{amsmath}
\usepackage{amsfonts}
\usepackage{graphicx}
\usepackage{pgfplots}
\pgfplotsset{compat=1.18}
\usepackage{enumitem}
\usepackage{hyperref}

\title{College Algebra Worksheet 6.4}
\author{PCL Learning Center}
\date{}

\begin{document}
\maketitle

\section*{Problem Set 1\\Difficulty level: Normal}
\subsection*{Problem 1}
What is the domain of the function
\[
g(x) = \ln\bigl(25x - x^2\bigr)?
\]
Give your answer in interval notation.

\subsection*{Problem 2}
Sketch by hand the following function.
\[
f(x) = -\log_{3}\bigl(x - 2\bigr),
\]


\subsection*{Problem 3}
What is the domain of the function
\[
h(x) = \log_{4}\bigl(x + 2\bigr) - 1?
\]
Give your answer in interval notation.

\subsection*{Problem 4}
What is the domain of each of the following functions? Give your answers in interval notation.
\begin{enumerate}[label=(\alph*)]
  \item \(g(x) = \ln\bigl(9x - x^2\bigr)\)
  \item \(h(x) = \log_{4}\bigl(x + 2\bigr) - 1\)
\end{enumerate}

\subsection*{Problem 5}
Sketch by hand the following function.
\[
f(x) = \log_{3}\bigl(x + 4\bigr) - 1,
\]

\subsection*{Problem 6}
Sketch by hand the following function.
\[
f(x) = \log_{2}(x - 1) + 3,
\]

\subsection*{Problem 7}
What is the domain of the function
\[
g(x) = \ln(4x - x^2)?
\]
Give your answer in interval notation.

\subsection*{Problem 8}
What is the domain of each of the following functions? Give your answers in interval notation.
\begin{enumerate}[label=(\alph*)]
  \item \( f(x) = \log_{5}(x - 3) + 2 \)
  \item \( g(x) = \ln(16x - x^2) \)
\end{enumerate}

\newpage
\section*{Solutions to the Set 1}

\subsection*{Problem 1}
Solve \(25x - x^2 > 0\):

\noindent\(25x - x^2 = x(25 - x) > 0 \quad\Rightarrow\quad (0,25)\)

\noindent Domain: \(\boxed{(0,25)}\)

\subsection*{Problem 2}
Plot the point: \((3,0)\).  

\noindent Vertical asymptote: \(x=2\).  

\noindent Domain: \(\boxed{(2,\infty)}\)  

\noindent Range: \(\boxed{(-\infty,\infty)}\)

\subsection*{Problem 3}
Solve \(x + 2 > 0\):

\noindent\(x > -2\)

\noindent Domain: \(\boxed{(-2,\infty)}\)

\subsection*{Problem 4}
\begin{enumerate}[label=(\alph*)]
  \item Solve \(9x - x^2 > 0\):

\noindent\(9x - x^2 = x(9 - x) > 0 \quad\Rightarrow\quad (0,9)\)

\noindent Domain: \(\boxed{(0,9)}\)

  \item Solve \(x + 2 > 0\):

\noindent\(x > -2\)

\noindent Domain: \(\boxed{(-2,\infty)}\)
\end{enumerate}

\subsection*{Problem 5}
Plot the point: \((-3,-1)\).

\noindent Vertical asymptote: \(x=-4\).

\noindent Domain: \(\boxed{(-4,\infty)}\)  

\noindent Range: \(\boxed{(-\infty,\infty)}\)

\subsection*{Problem 6}
Plot the point: \((2,3)\).

\noindent Vertical asymptote: \(x=1\).

\noindent Domain: \(\boxed{(1,\infty)}\)  

\noindent Range: \(\boxed{(-\infty,\infty)}\)

\subsection*{Problem 7}
Solve \(4x - x^2 > 0\):

\noindent\(4x - x^2 = x(4 - x) > 0 \quad\Rightarrow\quad (0,4)\)

\noindent Domain: \(\boxed{(0,4)}\)

\subsection*{Problem 8}
\begin{enumerate}[label=(\alph*)]
  \item Solve \(x - 3 > 0\):

\noindent\(x > 3\)

\noindent Domain: \(\boxed{(3,\infty)}\)

  \item Solve \(16x - x^2 > 0\):

\noindent\(16x - x^2 = x(16 - x) > 0 \quad\Rightarrow\quad (0,16)\)

\noindent Domain: \(\boxed{(0,16)}\)
\end{enumerate}

\end{document}
