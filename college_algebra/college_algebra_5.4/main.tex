\documentclass[12pt]{article}
\usepackage[a4paper, total={5.5in, 9in}]{geometry}
\usepackage{amsmath}
\usepackage{amsfonts}
\usepackage{graphicx}
\usepackage{pgfplots}
\pgfplotsset{compat=1.18}
\usepackage{enumitem}
\usepackage{hyperref}

\title{College Algebra Worksheet 5.4}
\author{PCL Learning Center}
\date{}

\begin{document}
\maketitle

\section*{Problem Set 1\\Difficulty level: Normal}
\subsection*{Problem 1} Consider the function \(f(x)=x^3+4x^2-25x-50\).\\
If there is a remainder of 50 when the function is divided by \((x-a)\), what is the value of \(a\)?
\subsection*{Problem 2} Consider the function \(f(x)=x^3+3x^2-51x+91\).\\
What is the remainder if \(f(x)\) is divided by \((x+9)\)?
\subsection*{Problem 3}
Use synthetic division to find the result when \(x^3+9x^2-11x-13\) is divided by \((x+5)\). Write your answer in the form \(q(x)+\dfrac{r(x)}{d(x)}\), where \(q(x)\) is the quotient, \(r(x)\) is the remainder, and \(d(x)\) is the divisor.

\subsection*{Problem 4}
Use synthetic division to find the result when \(2x^4-12x^2+5x-46\) is divided by \((x+4)\).\\
Write your answer in the form \(q(x)+\dfrac{r(x)}{d(x)}\), where \(q(x)\) is the quotient, \(r(x)\) is the remainder, and \(d(x)\) is the divisor.

\subsection*{Problem 5} Using synthetic division, what is the quotient for \((4x^3-x^2-2x+27)\div(x+2)\)?

\subsection*{Problem 6}
Using the remainder theorem, which binomial divisor of \(f(x)\) results in a remainder of 30 for the function \(f(x)=x^3-2x^2-93x+300\)?
\begin{enumerate}
    \item[(a)] \((x-3)\)
    \item[(b)] \((x-2)\)
    \item[(c)] \((x-1)\)
    \item[(d)] \((x+3)\)
    \item[(e)] \((x+2)\)
    \item[(f)] \((x+1)\)
\end{enumerate}

\subsection*{Problem 7}
Using the Remainder Theorem, find the remainder when \( f(x) = 2x^4 - 7x^3 + 5x^2 - 11x + 8 \) is divided by \( (x - 2) \).

\subsection*{Problem 8}
Consider the polynomial \( f(x) = x^3 - 6x^2 + 11x - 6 \). Use synthetic division to:
\begin{enumerate}[label=(\alph*)]
    \item Determine whether \( (x - 1) \) is a factor of \( f(x) \)
    \item If it is a factor, write the polynomial in factored form
    \item If not, find the remainder when \( f(x) \) is divided by \( (x - 1) \)
\end{enumerate}

\section*{Problem Set 2\\Difficulty level: Hard}
\subsection*{Problem 1}
Use the given length and area of a rectangle to express the width algebraically. Length is \(x+5\), area is \(2x^2+9x-5\).

\subsection*{Problem 2}
Use the given volume of a box and its length and width to express the height of the box algebraically. The volume is \(12x^3 + 20x^2 - 21x - 36\), the length is \(2x + 3\), and the width is \(3x - 4\).

\newpage
\section*{Solutions to the Set 1}
\subsection*{Problem 1} \(-5\)
\subsection*{Problem 2} 64
\subsection*{Problem 3} \(x^2+4x-31+\dfrac{142}{x+5}\)
\subsection*{Problem 4} \(2x^3-8x^2+20x-75+\dfrac{254}{x+4}\)
\subsection*{Problem 5} \(q(x)=4x^2-9x+16\)
\subsection*{Problem 6}
\begin{enumerate}
    \item[(a)] \((x-3)\) 
\end{enumerate}
\subsection*{Problem 7} \(-18\)

\subsection*{Problem 8}
\begin{enumerate}[label=(\alph*)]
    \item Yes, \((x-1)\) is a factor.
    \item \(f(x) = (x-1)(x^2-5x+6)=(x-1)(x-2)(x-3)\).
    \item Not applicable, since the remainder is 0.
\end{enumerate}

\section*{Solutions to the Set 2}
\subsection*{Problem 1} 
\(\text{Width}=\dfrac{\text{Area}}{\text{Length}}=\dfrac{2x^2+9x-5}{x+5}=2x-1\)
\subsection*{Problem 2}
\(\text{Height}=\dfrac{\text{Volume}}{\text{Length \(\cdot\) Width}}=\dfrac{12x^3+20x^2-21x-36}{(2x+3)(3x-4)}=2x+3\)


\end{document}
