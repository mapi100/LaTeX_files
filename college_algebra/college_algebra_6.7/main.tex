\documentclass[12pt]{article}
\usepackage[a4paper, total={5.5in, 9in}]{geometry}
\usepackage{amsmath}
\usepackage{amsfonts}
\usepackage{graphicx}
\usepackage{pgfplots}
\pgfplotsset{compat=1.18}
\usepackage{enumitem}
\usepackage{hyperref}

\title{College Algebra Worksheet 6.7}
\author{PCL Learning Center}
\date{}

\begin{document}
\maketitle

\section*{Problem Set 1\\Difficulty level: Normal}

\subsection*{Problem 1}
A freezer has a temperature of \(14\) degrees Fahrenheit. An ice-cube tray filled with water is placed in the freezer. The temperature after \(t\) minutes is modeled by
\[
f(t)=Ce^{-kt}+14,\quad k=0.045
\]
After 15 minutes, the water's temperature is \(40\) degrees. Find the initial temperature, rounded to the nearest degree.

\subsection*{Problem 2}
A population of kangaroos grows continuously at a rate of \(2\%\) per year, compounded continuously. How long (in years) will it take the population to reach \(150\%\) of its current size? Round your answer up to the nearest whole number.

\subsection*{Problem 3}
A state park had a population of \(330\) rabbits. The population grew with continuous compounding, at a rate of \(15\%\) per year for \(12\) years. How many rabbits are there after \(12\) years? Round down to the nearest whole number.

\subsection*{Problem 4}
A sample of bacteria grows continuously at an hourly rate of \(10\%\). Initially, there are \(11\) bacteria. How many bacteria will be present after \(18\) hours? Round your answer down to the nearest whole number.

\subsection*{Problem 5}
The population of a hornet nest grows continuously at a monthly rate of \(20\%\). After \(5\) months, there are \(815\) hornets. How many hornets were there initially? Assume each month has \(30\) days. Round your answer to the nearest whole number.

\subsection*{Problem 6}
A bank account is growing with continuous compounding at a fixed annual interest rate. The balance of the bank account doubles in \(6\) years. Which formula can determine the interest rate?
\begin{enumerate}[label=(\alph*)]
    \item \(2=r-1\)
    \item \(r=\dfrac{\ln(2)}{6}\)
    \item \(r(\ln(2))=6\)
    \item \(r=\dfrac{1}{2}\)
    \item \(\ln(6)=\dfrac{2}{r}\)
\end{enumerate}

\subsection*{Problem 7}
A large object with initial temperature \(150^\circ\)F is dropped into ocean water at \(77^\circ\)F. Its temperature after \(t\) minutes is modeled by
\[
f(t)=Ce^{-kt}+77
\]
After \(10\) minutes, the object is \(120^\circ\)F. After how many minutes will it reach \(80^\circ\)F? Round your answer to the nearest whole number.

\subsection*{Problem 8}
A medication has a half-life. Initially, there are \(20\) mg in the patient's system. After \(8\) hours, there are \(12\) mg. How many milligrams will remain after \(10\) hours? Round to the nearest hundredth.

\subsection*{Problem 9}
A bacteria sample decays according to a half-life model. Initially, there are \(700\) bacteria. After \(17\) minutes, there are \(420\) bacteria. After how many minutes will there be \(45\) bacteria? Round your answer to the nearest whole number.

\subsection*{Problem 10}
A cooler has a temperature of \(32^\circ\)F. A bottled drink is placed inside at an initial temperature of \(70^\circ\)F. The temperature after \(t\) minutes is modeled by
\[
f(t)=Ce^{-kt}+32
\]
After \(3\) minutes, the bottle has a temperature of \(42^\circ\)F. What is the approximate value of \(k\)? Round your answer to three decimal places.


\newpage
\section*{Solutions to the Set 1}

\subsection*{Problem 1}
\noindent Initial temperature: \(\boxed{65}\)

\subsection*{Problem 2}
\noindent Time: \(\boxed{21 \text{ years}}\)

\subsection*{Problem 3}
\noindent Number of rabbits: \(\boxed{1996}\)

\subsection*{Problem 4}
\noindent Bacteria after 18 hours: \(\boxed{66}\)

\subsection*{Problem 5}
\noindent Initial hornets: \(\boxed{300}\)

\subsection*{Problem 6}
\noindent Correct formula: \(\boxed{r=\dfrac{\ln(2)}{6}}\)

\subsection*{Problem 7}
\noindent Time to reach \(80^\circ\text{F}\): \(\boxed{60\text{ min}}\)

\subsection*{Problem 8}
\noindent Medication after 10 hours: \(\boxed{10.56}\)

\subsection*{Problem 9}
\noindent Time to reach 45 bacteria: \(\boxed{91\text{ min}}\)

\subsection*{Problem 10}
\noindent Value of \(k\): \(\boxed{0.445}\)

\end{document}
