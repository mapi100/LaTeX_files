\documentclass[12pt]{article}
\usepackage[a4paper, total={5.5in, 9in}]{geometry}
\usepackage{amsmath} % For mathematical formatting
\usepackage{changepage}
\usepackage[most]{tcolorbox}
\usepackage{textcomp} % for writing degrees

\title{Precalculus Worksheet 5.3}
\author{PCL Learning Center}
\date{}

\begin{document}
\maketitle

\begin{center}
    \textit{note: No graphing calculators or electronic devices may be used on this worksheet.}    
\end{center}

\section*{Problem Set 1\\Difficulty level: Normal}
\subsection*{Problem 1}
If \(\cos(t)=-\dfrac{3}{4}\) and \(t\) is in the second quadrant, find the exact value of each of the following trigonometric functions:
\[\tan(t),\cot(t),\sec(t),\csc(t)\]

\subsection*{Problem 2}
If \(\sin(t)=-\dfrac{1}{6}\), what is the exact value of \(\sin(-t)\)?\\

\subsection*{Problem 3}
If \(\tan(t) \approx 1.73 \) and \(\cos(t) \approx 0.31\), use the identities to find \(\sin(t)\) up to two decimal places.

\subsection*{Problem 4}
If \(\sec(t)=\dfrac{13}{5}\) and \(\sin(t)<0\), find the exact values of the following trigonometric functions:
\[\cos(t),\sin(t),\tan(t),\cot(t),\csc(t)\]

\newpage
\section*{Problem Set 2\\Difficulty level: Hard}
\subsection*{Problem 1}
Tangent and cotangent have a period of \(\pi\). What does this tell us about the output of these functions?

\subsection*{Problem 2}
On an interval of \([0,2\pi)\), can the sine and cosine values of a radian measure ever be equal? If so, where?

\subsection*{Problem 3}
Describe the secant function.

\newpage
\section*{Solutions for Set 1}
\subsection*{Problem 1}
\(\tan(t)=-\dfrac{\sqrt{7}}{3},\cot(t)=-\dfrac{3\sqrt{7}}{7},\sec(t)=-\dfrac{4}{3},\csc(t)=\dfrac{4\sqrt{7}}{7}\)
\subsection*{Problem 2}
\(\dfrac{1}{6}\)
\subsection*{Problem 3}
0.54
\subsection*{Problem 4}
\(\dfrac{5}{13},-\dfrac{12}{13},-\dfrac{12}{5},-\dfrac{5}{12},-\dfrac{13}{12}\)
\section*{Solutions for Set 2}
\subsection*{Problem 1}
The fact that \(\tan(t)\) and \(\cot(t)\) have a period of \(\pi\), means that their outputs repeat every \(\pi\) radians.
\subsection*{Problem 2}
Yes. If we assume that \(\sin(t)=\cos(t)\), then
\[\implies \tan(t)=1\]
As a solution, we have when \(t=\dfrac{\pi}{4}\) and \(t=\dfrac{5\pi}{4}\), respectively. These are the angles where the sine and cosine function are equal. \textit{Example: \(\sin(\pi/4)=\cos(\pi/4)\)}
\subsection*{Problem 3}
The secant function, is defined as the reciprocal of the cosine function. Hence, we have the following properties:
\begin{itemize}
    \item \textbf{Period}: \(2\pi\).
    \item \textbf{Domain}: All real numbers except where \(\cos(t)=0\).
    \item \textbf{Range}: \((-\infty,-1] \cup [1,\infty)\).
\end{itemize}
\end{document}
