\documentclass[12pt]{article}
\usepackage[a4paper, total={5.5in, 9in}]{geometry}
\usepackage{amsmath} % For mathematical formatting
\usepackage{changepage}
\usepackage[most]{tcolorbox}
\usepackage{textcomp} % for writing degrees
\usepackage{tikz} % For drawing the triangle
\usepackage{pgfplots}
\pgfplotsset{compat=1.18}
\usepackage{amsfonts}

\title{Precalculus Worksheet 8.1}
\author{PCL Learning Center}
\date{}

\begin{document}
\maketitle
\begin{center}
    \textit{note: No electronic devices may be used on this worksheet.}    
\end{center}

\section*{Problem Set 1\\Difficulty level: Normal}

\subsection*{Problem 1}
In \(\triangle ABC\), \(\sin A = \dfrac{1}{4}\), \(\sin C = \dfrac{2}{3}\), and \(a = 11\). Find the side length \(c\).

\subsection*{Problem 2}
In \(\triangle ABC\), \(\sin A = \dfrac{10}{3}\), \(\sin C = \dfrac{3}{7}\), and \(a = 9\). Find the side length \(c\).  


\subsection*{Problem 3}
The unknown triangle \(\triangle ABC\) has angle \(A = 10^\circ\) and sides \(a = 7\) and \(c = 8\).  
How many solutions are there for triangle \(\triangle ABC\)?  
If there are infinitely many, write \(\infty\).

\subsection*{Problem 4}
In \(\triangle ABC\), \(A = 20^\circ\), \(a = 27\) and \(c = 26\). Which of these statements best describes angle \(C\)?  
Select the correct answer below:
\begin{enumerate}
    \item[\textbf{A.}] \(C\) must be acute.
    \item[\textbf{B.}] \(C\) must be obtuse.
    \item[\textbf{C.}] \(C\) can be either acute or obtuse.
    \item[\textbf{D.}] \(\triangle ABC\) cannot be constructed.
\end{enumerate}

\subsection*{Problem 5}
If \(\angle A = 25^\circ\), \(a = 11\), \(b = 8\), solve the triangle \(\triangle ABC\). Round the measure of angles to the nearest whole degree and the measure of sides to the nearest tenth.

\subsection*{Problem 6}
If \(\angle A = 37^\circ\), \(a = 4\), \(b = 7\), solve the triangle \(\triangle ABC\). Round the measure of angles to the nearest whole degree and the measure of sides to the nearest tenth.

\subsection*{Problem 7}
When can you use the Law of Sines to find a missing angle?

\section*{Problem Set 2\\Difficulty level: Hard}
\subsection*{Problem 1}
In the Law of Sines, what is the relationship between the angle in the numerator and the side in the denominator?

\newpage
\section*{Solutions for the Set 1}
\subsection*{Problem 1}
88
\subsection*{Problem 2}
\(c=81/70\)
\subsection*{Problem 3}
1 solution.
\subsection*{Problem 4}
\begin{enumerate}
    \item[\textbf{A.}] \(C\) must be acute.
\end{enumerate}
\subsection*{Problem 5}
\(\angle B=18^{\circ}, \hspace{0.2cm} \angle C=137^{\circ}, \hspace{0.2cm} c=17.8\)
\subsection*{Problem 6}
With the given measurements, a triangle does not exist.
\subsection*{Problem 7}
You can use the Law of Sines to find a missing angle when you know either two sides and one non-included angle, or two angles and one side.

\section*{Solutions for the Set 2}
\subsection*{Problem 1}
We have,
\[\dfrac{\sin A}{a}=\dfrac{\sin B}{b}=\dfrac{\sin C}{c}\]
Each angle like \(A\) is in the numerator, and its opposite side like \(a\) is in the denominator. This relationship reflects how in any triangle, the sine of an angle is proportional to the length of the side opposite that angle.
 (i.e., the larger the angle in a triangle, the longer its opposite side, and vice versa).

\end{document}
