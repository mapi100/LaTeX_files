\documentclass[12pt]{article}
\usepackage[a4paper, total={5.5in, 9in}]{geometry}
\usepackage{amsmath} % For mathematical formatting
\usepackage{changepage}
\usepackage[most]{tcolorbox}
\usepackage{textcomp} % for writing degrees

\title{Precalculus Worksheet 5.2}
\author{PCL Learning Center}
\date{}

\begin{document}
\maketitle

\begin{center}
    \textit{note: No graphing calculators or electronic devices may be used on this worksheet.}    
\end{center}

\section*{Problem Set 1\\Difficulty level: Normal}
\subsection*{Problem 1}
Use the given sign of the sine and cosine functions to find the quadrant in which the terminal point of angle \(t\) lies.
\[\sin(t)>0 \hspace{0.2cm}\text{ and }\hspace{0.2cm} \cos(t)<0\]

\subsection*{Problem 2}
Angle \(t\) has the following sine and cosine values. Determine the quadrant in which the angle \(t\) lies.
\[\sin(t)=-0.681 \hspace{0.2cm}\text{ and }\hspace{0.2cm} \cos(t)=-0.257\]

\subsection*{Problem 3}
State the reference angle for the given angle: \(294^{\circ}\).

\subsection*{Problem 4}
State the reference angle for the given angle: \(-\dfrac{7\pi}{4}\).\\

\subsection*{Problem 5}
If \(\cos(t)=-\dfrac{2}{5}\) and angle \(t\) is in the third quadrant, find the exact value of \(\sin(t)\).\\

\subsection*{Problem 6}
If \(\sin(t)=-\dfrac{4}{9}\) and angle \(t\) is in the fourth quadrant, find the exact value of \(\cos(t)\).\\

\section*{Problem Set 2\\Difficulty level: Hard}
\subsection*{Problem 1}
Find the exact value of the following trigonometric function.
\[\sin\Big(\arccos\Big(-\dfrac{4}{5}\Big)\Big)\]

\subsection*{Problem 2}
Find the exact value of the following trigonometric function.
\[\cos\Big(\arcsin\Big(\dfrac{5}{13}\Big)\Big)\]

\newpage
\section*{Solutions for Set 1}
\subsection*{Problem 1}
\(\sin>0\) in QI/QII, \(\cos<0\) in QII/QIII. Intersection: \(\boxed{II}\)

\subsection*{Problem 2}
\(\sin<0\) in QIII/QIV, \(\cos<0\) in QII/QIII. Intersection: \(\boxed{III}\)

\subsection*{Problem 3}
\(294^\circ = 360^\circ - 66^\circ\). Reference angle: \(\boxed{66^\circ}\)

\subsection*{Problem 4}
\(-\dfrac{7\pi}{4} + 2\pi = \dfrac{\pi}{4}\). Reference angle: \(\boxed{\dfrac{\pi}{4}}\)

\subsection*{Problem 5}
Using Pythagorean Theorem,
\[
\sin(t) = -\sqrt{1 - \left(-\dfrac{2}{5}\right)^2} = -\sqrt{1 - \dfrac{4}{25}} = -\sqrt{\dfrac{21}{25}} = \boxed{-\dfrac{\sqrt{21}}{5}}
\]
(Negative in QIII)

\subsection*{Problem 6}
Using Pythagorean Theorem,
\[
\cos(t) = \sqrt{1 - \left(-\dfrac{4}{9}\right)^2} = \sqrt{1 - \dfrac{16}{81}} = \sqrt{\dfrac{65}{81}} = \boxed{\dfrac{\sqrt{65}}{9}}
\]
(Positive in QIV)

\section*{Solutions for Set 2}
\subsection*{Problem 1}
Let \(\theta = \arccos\left(-\dfrac{4}{5}\right)\). Then:
\[
\sin(\theta) = \sqrt{1 - \left(-\dfrac{4}{5}\right)^2} = \sqrt{\dfrac{9}{25}} = \boxed{\dfrac{3}{5}}
\]


\subsection*{Problem 2}
Let \(\theta = \arcsin\left(\dfrac{5}{13}\right)\). Then:
\[
\cos(\theta) = \sqrt{1 - \left(\dfrac{5}{13}\right)^2} = \sqrt{\dfrac{144}{169}} = \boxed{\dfrac{12}{13}}
\]

\end{document}