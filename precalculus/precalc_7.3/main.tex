\documentclass[12pt]{article}
\usepackage[a4paper, total={5.5in, 9in}]{geometry}
\usepackage{amsmath} % For mathematical formatting
\usepackage{changepage}
\usepackage[most]{tcolorbox}
\usepackage{textcomp} % for writing degrees
\usepackage{tikz} % For drawing the triangle
\usepackage{pgfplots}
\pgfplotsset{compat=1.18}

\title{Precalculus Worksheet 7.3}
\author{PCL Learning Center}
\date{}

\begin{document}
\maketitle

\begin{center}
    \textit{note: No graphing calculators or electronic devices may be used on this worksheet.}    
\end{center}

\section*{Problem Set 1\\Difficulty level: Normal}

\subsection*{Problem 1}
Given that \(\tan(\theta)=3\), what is \(\tan(2\theta)\)?

\subsection*{Problem 2}
For all values of \(\theta\) for which the expression is defined,
\[\dfrac{\cos(2\theta)}{\tan^2\theta}=?\]
    \begin{enumerate}
    \item[(a)] \(2\cot^2\theta - \csc^2\theta\)
    \item[(b)] \(2\cos\theta - \sec\theta\)
    \item[(c)] \(2\sin\theta\)
    \item[(d)] \(\cot\theta - 2\sin\theta\cos\theta\)
    \item[(e)] \(\cot^2\theta - 2\cos^2\theta\)
    \end{enumerate}
    
\subsection*{Problem 3}
Use a half-angle formula to find \(\tan\left(\dfrac{7\pi}{12}\right)\).

\subsection*{Problem 4}
Given that \(\sin(\theta)=\dfrac{17}{19}\), and \(\theta\) is in Quadrant \(II\), what is \(\cos(2\theta)\)?

\subsection*{Problem 5}
Given that \(\sin(\theta)=-\dfrac{12}{13}\), and \(\theta\) is in Quadrant \(III\), what is \(\sin(2\theta)\)?

\subsection*{Problem 6}
Use a half-angle formula to find \(\sin(67.5^\circ)\).

\subsection*{Problem 7}
If \(\cos(\theta)=\dfrac{24}{25}\), and \(\theta\) is in Quadrant \(I\), then what is \(\cos\left(\dfrac{\theta}{2}\right)\)?

\section*{Problem Set 2\\Difficulty level: Hard}
\subsection*{Problem 1}
Explain how to determine the double-angle formula for \(\tan(2x)\) using the double-angle formulas for \(\cos(2x)\) and \(\sin(2x)\).

\subsection*{Problem 2}
Prove the identities of the following trigonometric functions.
\begin{enumerate}
    \item[(a)] \((\sin \alpha-\cos \alpha)^2 = 1 - \sin(2\alpha)\)
    \item[(b)] \(\sin(2\beta) = -2\sin(-\beta)\cos(-\beta)\)
    \item[(c)] \(\dfrac{1+\cos(2\theta)}{\sin(2\theta)}\tan^2\theta = \tan \theta\)
\end{enumerate}

\newpage
\section*{Solutions for Set 1}
\subsection*{Problem 1}
\(\tan(2\theta)=-3/4\)
\subsection*{Problem 2}
(e) \(\cot ^2 \theta -2\cos ^2 \theta\)
\subsection*{Problem 3}
\(-2-\sqrt{3}\)
\subsection*{Problem 4}
\(\cos(2\theta)=-217/361\)
\subsection*{Problem 5}
\(\sin(2\theta)=120/169\)
\subsection*{Problem 6}
\(\dfrac{\sqrt{2+\sqrt{2}}}{2}\)
\subsection*{Problem 7}
\(\cos\Big(\dfrac{\theta}{2}\Big)=\dfrac{7\sqrt{2}}{10}\)

\section*{Solutions for Set 2}
\subsection*{Problem 1}
\begin{align*}
    \tan(2x)&=\dfrac{\sin(2x)}{\cos(2x)}\\
    &=\dfrac{2\sin x \cos x}{\cos^2x-\sin^2x}\\
    &=\dfrac{2\tan x}{1-\tan^2x}
\end{align*}

\subsection*{Problem 2}
\begin{enumerate}
    \item[(a)] \begin{align*}
        (\sin \alpha - \cos \alpha)^2 &= 1 - \sin(2\alpha) \\
        \sin^2 \alpha - 2\sin \alpha \cos \alpha + \cos^2 \alpha &= \\
        \sin^2 \alpha + \cos^2 \alpha - 2\sin \alpha \cos \alpha &= \\
        1 - \sin(2\alpha) &=
    \end{align*}

    \item[(b)] \begin{align*}
        \sin(2\beta) &= -2\sin(-\beta)\cos(-\beta) \\
        &= -2(-\sin \beta)(\cos \beta) \\
        &= 2\sin \beta \cos \beta \\
        &= \sin(2\beta)
    \end{align*}

    \item[(c)] \begin{align*}
    \dfrac{1+\cos(2\theta)}{\sin(2\theta)}\tan^2\theta &= \dfrac{1 + (\cos^2\theta - \sin^2\theta)}{2\sin\theta\cos\theta} \cdot \dfrac{\sin^2\theta}{\cos^2\theta} \\
    &= \dfrac{(1 + \cos^2\theta - \sin^2\theta)\sin^2\theta}{2\sin\theta\cos^3\theta} \\
    &= \dfrac{(2\cos^2\theta)\sin^2\theta}{2\sin\theta\cos^3\theta} \\
    &= \dfrac{2\cos^2\theta \cdot \sin\theta}{2\cos^3\theta} \\
    &= \dfrac{\sin\theta}{\cos\theta} = \tan\theta
\end{align*}
\end{enumerate}




\end{document}
