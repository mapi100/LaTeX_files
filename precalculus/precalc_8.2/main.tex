\documentclass[12pt]{article}
\usepackage[a4paper, total={5.5in, 9in}]{geometry}
\usepackage{amsmath} % For mathematical formatting
\usepackage{changepage}
\usepackage[most]{tcolorbox}
\usepackage{textcomp} % for writing degrees
\usepackage{tikz} % For drawing the triangle
\usepackage{pgfplots}
\pgfplotsset{compat=1.18}
\usepackage{amsfonts}

\title{Precalculus Worksheet 8.2}
\author{PCL Learning Center}
\date{}

\begin{document}
\maketitle

\begin{center}
    \textit{note: No electronic devices may be used on this worksheet.}    
\end{center}

\section*{Problem Set 1\\Difficulty level: Normal}
\subsection*{Problem 1}
For the exercises you see below, assume \(\alpha\) is opposite side \(a\), \(\beta\) is opposite side \(b\), and \(\gamma\) is opposite side \(c\). If possible, solve each triangle for the unknown side. Round to the nearest tenth.

    \begin{itemize}
        \item[(i)] \(\gamma = 41.2^{\circ}, a=2.49, b=3.13\)
        \item[(ii)] \(\alpha=120^{\circ},b=6,c=7\)
        \item[(iii)] \(\beta=58.7^{\circ},a=10.6,c=15.7\)
        \item[(iv)] \(\gamma=115^{\circ},a=18,b=23\)
    \end{itemize}

\subsection*{Problem 2}
Given \(\triangle ABC\) below, with \(\angle C=20^{\circ}, \hspace{0.2cm} a=11,\) and \(b=5.9\), find the area of the triangle.

\subsection*{Problem 3}
Use Heron's formula to find the area of the triangle with side lengths \(5,12,\) and \(16,\) as shown below.

\subsection*{Problem 4}
Use Heron's formula to find the area of the triangle with side lengths \(43,50,\) and \(56,\) as shown below.

\subsection*{Problem 5}
For the following exercises, use the Law of Cosines to solve for the missing angle of the oblique triangle. Round to the nearest tenth.

   \begin{itemize}
        \item[(i)] \(a=42,b=19,c=30;\) find \(\angle A\)
        \item[(ii)] \(a=13,b=22,c=28;\) find \(\angle A\)
    \end{itemize}

\section*{Problem Set 2\\Difficulty level: Hard}
\subsection*{Problem 1}
Explain the relationship between the Pythagorean Theorem and the Law of Cosines.

\newpage
\section*{Solutions for the Set 1}
\subsection*{Problem 1}
    \begin{itemize}
        \item[(i)] \(c \approx 2.1, \alpha \approx 52.5^{\circ}, \beta \approx 86.3^{\circ} \)
        \item[(ii)] \(a \approx 11.3, \beta \approx 27.5^{\circ}, \gamma \approx 32.5^{\circ} \)
        \item[(iii)] \(b \approx 13.6, \alpha \approx 41.6^{\circ}, \gamma \approx 79.7^{\circ} \)
        \item[(iv)] \(c \approx 34.7, \alpha \approx 28.1^{\circ}, \beta \approx 36.9^{\circ} \)
    \end{itemize}
\subsection*{Problem 2}
\(\text{Area} \approx 11.1\)
\subsection*{Problem 3}
\(\text{Area} \approx 20.7\)
\subsection*{Problem 4}
\(\text{Area} \approx 1031.3\)

\subsection*{Problem 5}
   \begin{itemize}
        \item[(i)] \(\angle A \approx 116.2^{\circ}\)
        \item[(ii)] \(\angle A \approx 26.9^{\circ}\)
    \end{itemize}

\section*{Solutions for the Set 2}
\subsection*{Problem 1}
The Law of Cosines extends the Pythagorean Theorem to all triangles. When the angle between two sides is \(90^{\circ}\), the Law of Cosines simplifies to the Pythagorean Theorem.

\end{document}
