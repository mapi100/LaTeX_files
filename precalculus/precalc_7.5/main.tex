\documentclass[12pt]{article}
\usepackage[a4paper, total={5.5in, 9in}]{geometry}
\usepackage{amsmath} % For mathematical formatting
\usepackage{changepage}
\usepackage[most]{tcolorbox}
\usepackage{textcomp} % for writing degrees
\usepackage{tikz} % For drawing the triangle
\usepackage{pgfplots}
\pgfplotsset{compat=1.18}
\usepackage{amsfonts}

\title{Precalculus Worksheet 7.5}
\author{PCL Learning Center}
\date{}

\begin{document}
\maketitle

\begin{center}
    \textit{note: This worksheet may be used for two sessions, and\\no electronic devices may be used on this worksheet.}
\end{center}

\section*{Problem Set 1\\Difficulty level: Normal}

\subsection*{Problem 1}
Solve the following equation for \(\theta\) on the interval \([0,2\pi).\)
\[-7\sqrt{3}\tan \theta + 2 = 9\]

\subsection*{Problem 2}
Solve the following trigonometric equation for \(0 \leq \theta < 2\pi\).
\[(\csc \theta - 2)(\cot \theta + 1) = 0\]

\subsection*{Problem 3}
Determine the exact value of \(\theta\) in the following equation if \(0 \leq \theta < 2\pi\).
\[-4\cos \theta + 1 = 5\]

\subsection*{Problem 4}
If \(0 \leq \theta < 2\pi\) and \(-14\sin \theta - 7 = -7\), determine the values of \(\theta\).

\subsection*{Problem 5}
Solve the following equation for \(\theta\) on the interval \([0^\circ, 360^\circ)\).
\[6\sqrt{2} \sec(\theta) + 4 = 16\]

\subsection*{Problem 6}
Solve the following equation for \(\theta\) on the interval \([90^\circ, 270^\circ)\).
\[3\sin \theta + 2 = 0\]

\subsection*{Problem 7}
Solve the following equation for \(\beta\), where \(0 \leq \beta < 2\pi\).
\[6\cos^2 \beta - 5\cos \beta = 4\]

\subsection*{Problem 8}
Determine the exact value of \(\theta\) in the following equation if \(0 \leq \theta < 2\pi\).
\[-8\cos^2 \theta + 6 = 2\]

\subsection*{Problem 9}
Solve the equation below for \(\theta\), where \(0 \leq \theta < 2\pi\).
\[-8\sin^2 \theta - 4 = -10\]

\subsection*{Problem 10}
If \(-\pi \leq \theta < \pi\), find all values of \(\theta\) that satisfy the equation below.
\[4\tan^2 \theta = 4\tan \theta\]

\section*{Problem Set 2\\Difficulty level: Hard}
\subsection*{Problem 1}
Will there always be solutions to trigonometric function equations? If not, describe an equation that would not have a solution. Explain why or why not.

\subsection*{Problem 2}
Write the general solution for the following trigonometric function. (i.e., all solutions for the interval \((-\infty,\infty)\).
\[\dfrac{\sin(2x)}{2\csc^2x}=0\]

\newpage
\section*{Solutions for the Set 1}
\subsection*{Problem 1}
\(\theta=\dfrac{5\pi}{6}\) and \(\theta=\dfrac{11\pi}{6}\)
\subsection*{Problem 2}
\(\dfrac{\pi}{6},\dfrac{3\pi}{4},\dfrac{5\pi}{6},\dfrac{7\pi}{4}\)
\subsection*{Problem 3}
\(\theta=\pi\)
\subsection*{Problem 4}
\(\theta=0\) and \(\theta=\pi\)
\subsection*{Problem 5}
\(\theta=45^{\circ}\) and \(\theta=315^{\circ}\)
\subsection*{Problem 6}
\(\theta \approx 221.8^{\circ}\)
\subsection*{Problem 7}
\(\beta=\dfrac{2\pi}{3}\) and \(\beta=\dfrac{4\pi}{3}\)
\subsection*{Problem 8}
\(\theta=\dfrac{\pi}{4}, \hspace{0.2cm} \dfrac{3\pi}{4}, \hspace{0.2cm} \dfrac{5\pi}{4}, \hspace{0.2cm} \dfrac{7\pi}{4}\)
\subsection*{Problem 9}
\(\theta= \dfrac{\pi}{3}, \hspace{0.2cm} \dfrac{2\pi}{3}, \hspace{0.2cm} \dfrac{4\pi}{3}, \hspace{0.2cm} \dfrac{5\pi}{3}\)
\subsection*{Problem 10}
\(\theta=-\pi, \hspace{0.2cm} -\dfrac{3\pi}{4}, \hspace{0.2cm} 0, \hspace{0.2cm} \dfrac{\pi}{4}\)

\section*{Solutions for the Set 2}
\subsection*{Problem 1}
Not all trigonometric equations have solutions. As an example, \(\sin x =2\) has no solution, since the sine function only takes values in the range \([-1,1]\).

\subsection*{Problem 2}
\(x=\pi n, \hspace{0.2cm} \) for all \(n \in \mathbb{Z}\)\\
\(x=\pi n-\pi/2, \hspace{0.2cm}\) for all \(n \in \mathbb{Z}\)\\
\textit{recall: \(\mathbb{Z}\) is the set of all integers}.
\end{document}
