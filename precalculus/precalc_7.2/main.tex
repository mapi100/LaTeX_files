\documentclass[12pt]{article}
\usepackage[a4paper, total={5.5in, 9in}]{geometry}
\usepackage{amsmath} % For mathematical formatting
\usepackage{changepage}
\usepackage[most]{tcolorbox}
\usepackage{textcomp} % for writing degrees
\usepackage{tikz} % For drawing the triangle
\usepackage{pgfplots}
\pgfplotsset{compat=1.18}
\usepackage{hyperref}

\title{Precalculus Worksheet 7.2}
\author{PCL Learning Center}
\date{}

\begin{document}
\maketitle

\begin{center}
    \textit{note: No graphing calculators or electronic devices may be used on this worksheet.}    
\end{center}

\section*{Problem Set 1\\Difficulty level: Normal}

\subsection*{Problem 1}
You are given that \(\cos(A)=-\dfrac{4}{5}\), with \(A\) in Quadrant \(III\), and \(\cos(B)=\dfrac{15}{17}\), with \(B\) in Quadrant \(I\). Find \(\cos(A+B)\).

\subsection*{Problem 2}
You are given that \(\sin(A)=\dfrac{8}{17}\), with \(A\) in Quadrant \(I\), and \(\sin(B)=\dfrac{12}{13}\), with \(B\) in Quadrant \(II\). Find \(\sin(A+B)\).

\subsection*{Problem 3}
You are given that \(\tan(A)=\dfrac{11}{3}\) and \(\tan(B)=7\). Find \(\tan(A-B)\).

\subsection*{Problem 4}
You are given that \(\cos(A)=\dfrac{12}{13}\), with \(A\) in Quadrant \(I\), and \(\cos(B)=\dfrac{5}{13}\), with \(B\) in Quadrant \(I\). Find \(\cos(A-B)\).

\subsection*{Problem 5}
You are given that \(\cos(A)=-\dfrac{15}{17}\), with \(A\) in Quadrant \(III\), and \(\sin(B)=\dfrac{4}{5}\), with \(B\) in Quadrant \(II\). Find \(\sin(A-B)\).

\subsection*{Problem 6}
You are given that \(\tan(A)=2\) and \(\tan(B)=9\). Find \(\tan(A+B)\).

\section*{Problem Set 2\\Difficulty level: Hard}

\subsection*{Problem 1}
Is there only one way to evaluate \(\cos\Big(\dfrac{5\pi}{4}\Big)\)? Explain how to set up the solution in two different ways, and then compute to make sure they give the same answer.

\subsection*{Problem 2}
For the following trigonometric functions, find the exact value of each expression.
\begin{align*}
    &\sin\Big(\arccos(0)-\arccos\Big(\dfrac{1}{2}\Big)\Big)\\
    &\cos\Big(\arccos\Big(\dfrac{\sqrt{2}}{2}\Big)+\arcsin\Big(\dfrac{\sqrt{3}}{2}\Big)\Big)
\end{align*}

\subsection*{Problem 3}
Simplify the following trigonometric function and then graph both expressions as functions to verify the graphs are identical.
\begin{align*}
    &\tan\Big(\dfrac{\pi}{4}-x\Big)
\end{align*}


\newpage
\section*{Solutions for Set 1}
\subsection*{Problem 1}
\(-36/85\)
\subsection*{Problem 2}
\(140/221\)
\subsection*{Problem 3}
\(-1/8\)
\subsection*{Problem 4}
\(120/169\)
\subsection*{Problem 5}
\(84/85\)
\subsection*{Problem 6}
\(-11/17\)
\section*{Solutions for Set 2}
\subsection*{Problem 1}
By the unit circle, because \(5\pi/4\) lies in Quadrant \(III\), its reference angle is \(\pi/4\), and since we know that cosine is negative in Quadrant \(III\), we have
\[\cos\Big(\dfrac{5\pi}{4}\Big)=-\dfrac{\sqrt{2}}{2}\]
Also, using a known identity, notice that \(\dfrac{5\pi}{4}=\pi+\dfrac{\pi}{4}\). Hence,
\begin{align*}
\cos(\pi+x)&=-\cos(x)\\
\cos\Big(\dfrac{5\pi}{4}\Big)&=-\cos\Big(\dfrac{\pi}{4}\Big)=-\dfrac{\sqrt{2}}{2}
\end{align*}
\subsection*{Problem 2}
\begin{align*}
\sin\Big(\arccos(0)-\arccos\Big(\dfrac{1}{2}\Big)\Big)&=\dfrac{1}{2}\\
\cos\Big(\arccos\Big(\dfrac{\sqrt{2}}{2}\Big)+\arcsin\Big(\dfrac{\sqrt{3}}{2}\Big)\Big)&=-\dfrac{\sqrt{3}-1}{2\sqrt{2}}
\end{align*}

\subsection*{Problem 3}
\[\tan\Big(\dfrac{\pi}{4}-x\Big)=\dfrac{1-\tan(x)}{1+\tan(x)}\]
To see the graph, sketch \(\tan\Big(\dfrac{\pi}{4}-x\Big)=\dfrac{1-\tan(x)}{1+\tan(x)}\) on \href{https://www.desmos.com/calculator}{Desmos} or \href{https://www.geogebra.org/graphing?lang=en}{GeoGebra} to see the solution.



\end{document}
